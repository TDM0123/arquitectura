\documentclass{article}

\usepackage[spanish]{babel}
\usepackage[T1]{fontenc}
\usepackage[utf8]{inputenc}

\usepackage{marvosym}
\usepackage[margin = 1in]{geometry}
\usepackage{amsmath}
\usepackage{amssymb}
\usepackage{multirow}

\title{Fórmulas Primer Capítulo del Libro}
\author{Moises Del Rosario C.I: 29.626.161}
\date{\today}

\selectlanguage{spanish}

\begin{document}
\maketitle

\section{ \uppercase{Prestaciones del CPU} }

\subsection{Prestaciones de un computador:}

$$Prestaciones_{X} = \frac{1}{Tiempo\ de\ ejecucion\ de\ un\ programa_{X}}$$
\

\subsection{Comparación de prestaciones entre equipos:}

$$Prestaciones_{X} > Prestaciones_{Y}$$

$$\equiv$$

$$\frac{1}{Tiempo\ de\ ejecucion\ de\ un\ programa_{X}} > \frac{1}{Tiempo\ de\ ejecucion\ de\ un\ programa_{Y}}$$
\

$$\equiv$$

$$Tiempo\ de\ ejecucion\ de\ un\ programa_{Y} > Tiempo\ de\ ejecucion\ de\ un\ programa_{X}$$\

\subsection{Relación cuantitativa entre las prestaciones de dos computadores:}
$$\frac{Prestaciones_{X}}{Prestaciones_{Y}} = n$$

$$\equiv$$

$$\frac{Prestaciones_{X}}{Prestaciones_{Y}} = \frac{Tiempo\ de\ ejecucion\ de\ un\ programa_{Y}}{Tiempo\ de\ ejecucion\ de\ un\ programa_{X}} = n$$

\subsection{Medir prestaciones de un computador en base a otros factores:}

$$Tiempo\ de\ ejecucion\ de\ un\ programa = Ciclos\ del\ programa \ *\ El\ tiempo\ de\ un\ ciclo\ de\ reloj$$

$$\equiv$$

$$Tiempo\ de\ ejecucion\ de\ un\ programa = \frac{Ciclos\ del\ programa}{Frecuencia\ del\ reloj\ del\ computador}$$

\section{ \uppercase{Ciclos e instrucciones del CPU como factores de prestaciones} }

\subsection{Ciclos para un programa en base a prestaciones:}

$$Ciclos\ de\ reloj\ del\ CPU\ para\ un\ programa = N^o\ de\ instrucciones * CPI$$


\subsection{Ecuación clásica para medir las prestaciones de la CPU:}

$$Tiempo\ de\ ejecucion = {N^o\ de\ instrucciones} * {CPI} * {Tiempo\ de\ ciclo}$$

$$\equiv$$

$$Tiempo\ de\ ejecucion = \frac{N^o\ de\ instrucciones * CPI}{Frecuencia\ del\ reloj\ del\ computador}$$


\subsection{Ecuación elemental de prestaciones con factores simplificados:}

$$\frac{segundos}{instrucciones} = \frac{Instrucciones}{Programa} * \frac{Ciclos\ de\ reloj}{Instruccion} * \frac{Segundos}{Ciclos\ de\ reloj}$$

\section{ \uppercase{Potencia y su relación con las prestaciones - medir la potencia relativa de un procesador en relación con otro} }

$$\frac{Potencia_{nuevo}}{Potencia_{antiguo}} = \frac{Carga\ capacitiva_{nuevo} * Voltaje^2_{nuevo} * Frecuencia_{nuevo}}{Carga\ capacitiva_{antiguo} * Voltaje^2_{antiguo} * Frecuencia_{antiguo}}$$


\section{ \uppercase{Circuitos integrados y sus prestaciones de producción} }

\subsection{Coste por dado de silicio:}

$$coste\ por\ dado = \frac{coste\ por\ oblea}{dado\ por\ oblea * factor\ de\ produccion}$$

\subsection{Dados obtenidos por Oblea (lonja) de silicio:}

$$Dados\ por\ oblea = \frac{area\ de\ la\ oblea}{area\ del\ dado}$$

\subsection{Porcentaje de dados sin defectos por tanda(aproximación empírica:}

$$Factor\ de\ produccion =\frac{1}{(1 + (defectos\ por\ area * area\ del\ dado))^2}$$

\section{ \uppercase{Pruebas SPEC (Standard Performance Evaluation Corporation)} }

\subsection{Formula para medir prestaciones con la geométrica para razones SPEC:}

$$n\sqrt{\prod_{i\ =\ 1}^{n} Relaciones\ de\ tiempo\ de\ ejecucion}$$

$$n = Numero\ de\ programas\ en\ la\ carga\ de\ trabajo$$
$$RTE = Razon\ SPEC\ del\ tiempo\ del\ modelo\ de\ referencia\ entre\ el\ del\ modelo\ nuevo$$


\subsection{Formula SPEC para medir prestaciones según el nivel de potencia de un equipo:}

$$ssj\_ops\ global\ por\ vatio = \frac{\sum_{i\ =\ 0}^{10}\ ssj\_ops_{i}}{\sum_{i\ =\ 0}^{10}\ potencia_{i}}$$

\section{ \uppercase{Relación entre mejoras individuales y prestaciones en el computador (Ley de Amdahl)} }

$$Tiempo\ de\ ejecucion\ despues\ de\ la\ mejora = \frac{tiempo\ de\ ejecucion\ de\ la\ mejora}{la\ cantidad\ de\ mejoras} + el\ tiempo\ no\ afectado$$

\section{ \uppercase{MIPS(Millones de Instrucciones Por Segundo)} }

$$MIPS = \frac{N^o\ de\ instrucciones}{tiempo\ de\ ejecucion * 10^{6}}$$

\end{document}